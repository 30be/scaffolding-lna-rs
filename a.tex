\documentclass[10pt,twocolumn]{article}

% Compact 2-page layout (keeps all words; adjusts only formatting)
\usepackage[T2A]{fontenc}
\usepackage[utf8]{inputenc}
\usepackage[russian]{babel}

\usepackage[margin=0.6in]{geometry}
\usepackage{graphicx}
\usepackage{amsmath}
\usepackage{caption}
\usepackage{microtype}
\usepackage{titlesec}

\setlength{\parindent}{0pt}
\setlength{\parskip}{0.35\baselineskip}
\setlength{\columnsep}{18pt}
\emergencystretch=1em

% Tighten headings and floats to help meet the 2-page constraint
\titlespacing*{\section}{0pt}{0.7\baselineskip}{0.35\baselineskip}
\setlength{\abovecaptionskip}{2pt}
\setlength{\belowcaptionskip}{0pt}
\setlength{\textfloatsep}{6pt}
\setlength{\intextsep}{6pt}
\captionsetup{font=small,labelfont=bf}

\pagestyle{empty}

\begin{document}

\section*{Разработка высокопроизводительного алгоритмического комплекса для гуманизации антител на основе структурного сопоставления}

\textbf{Автор:} Софрыгин Лука

\textbf{Научный руководитель:} Блохин Никита (компания BIOCAD)

\section*{1. Актуальность и постановка задачи}

Создание современных лекарств на основе антител животных (например, лам или верблюдов) является одним из наиболее перспективных направлений биофармацевтики. Такие антитела обладают уникальными свойствами, однако их прямое использование в терапии человека невозможно из-за высокого риска иммунного отторжения. Для решения этой проблемы применяется процесс гуманизации — пересадка активных участков (CDR), отвечающих за связывание с мишенью, на человеческий белковый каркас.

Изначальный прототип системы, реализованный автором на языке Python, оказался недостаточно эффективным. Интерпретируемая природа языка привела к критическим задержкам при обработке массивов данных, содержащих тысячи белковых структур, затрудняя разработку и проверку гипотез.

Целью данной работы стало создание автономного высокопроизводительного программного комплекса \texttt{scaffolding-rs} на системном языке программирования Rust. Переход на компилируемый код и использование строгих алгоритмических методов сравнения пространственных структур позволили на несколько порядков ускорить процесс подбора каркасов при сохранении строгой биологической релевантности результатов.

\section*{2. Методы решения}

Ключевым этапом работы стал полный рефакторинг системы. Выбор языка \textbf{Rust} обусловлен необходимостью строгого управления памятью без использования сборщика мусора, гарантированной безопасностью многопоточных вычислений и высокой скоростью исполнения бинарного кода.

\subsection*{Технологические решения}

\begin{itemize}
  \item \textbf{Хранение данных}: Был осуществлен отказ от файловой системы в пользу встраиваемой базы данных \textbf{SQLite} в режиме опережающей записи (WAL). Это позволило хранить тысячи PDB-структур и их метаданные в одном компактном файле, обеспечивая эффективный параллельный доступ на чтение и запись без блокировок.
  \item \textbf{Параллелизм}: С использованием библиотеки \texttt{rayon} реализовано распараллеливание тяжелых геометрических вычислений (расчет двугранных углов, наложение структур) на все доступные ядра процессора, что обеспечило линейную масштабируемость производительности.
\end{itemize}

\subsection*{Алгоритмические решения}

\begin{enumerate}
  \item \textbf{Подготовка и очистка данных}: База данных SAbDab содержит структуры разного качества. Был разработан модуль валидации, который анализирует каждый файл на наличие пропусков в нумерации аминокислот и геометрических разрывов в белковой цепи (расстояние между атомами углерода и азота соседних остатков более 2.0 \AA). Структуры, не прошедшие проверку, помечаются и исключаются. В ходе обработки тестовой выборки было сохранено \textbf{90\%} наиболее достоверных структур (4500 из 5000), что исключило «шум» в результатах.

  \begin{center}
    \includegraphics[width=0.95\columnwidth]{pics/gap_analysis.png}

    \small\emph{Рис. 1. Распределение длины пептидной связи в базе. Выбросы справа от пунктирной линии (2.0А) указывают на физические разрывы цепи и отсеиваются алгоритмом.}
  \end{center}

  \item \textbf{Аннотация}: Для унификации позиций аминокислот внедрена схема нумерации Мартина. Интеграция с инструментом \textbf{ANARCII} реализована через изолированное программное окружение, что позволяет автоматически загружать и кэшировать модели машинного обучения.

  \item \textbf{Структурное сопоставление}: Поиск каркаса ведется не по последовательности, а по геометрической близости укладки остова белка. Используется метрика среднеквадратичного отклонения координат (СКО) атомов. Дополнительно проводится анализ двугранных углов (карта Рамачандрана), позволяющий оценивать стереохимическую корректность свернутой молекулы и отсеивать энергетически невыгодные конформации.

  \begin{center}
    \includegraphics[width=0.95\columnwidth]{pics/cdr_lengths.png}

    \small\emph{Рис. 2. Вариативность длин активных участков (CDR H3) в подготовленной базе данных, демонстрирующее разнообразие доступных структур.}
  \end{center}
\end{enumerate}

\section*{3. Анализ полученных результатов}

Разработанный комплекс \texttt{scaffolding-rs} демонстрирует качественное превосходство над прототипом:

\begin{enumerate}
  \item \textbf{Производительность}: Время полной обработки базы сократилось с нескольких часов до минут. Поиск совпадений для одной структуры теперь занимает доли секунды, что позволяет проводить массовый скрининг.
  \item \textbf{Надежность}: Проведено стресс-тестирование путем внесения случайных искажений (шума) в координаты атомов тестовой структуры. Алгоритм успешно находит исходную структуру среди тысяч других даже при наличии искажений, что подтверждает устойчивость выбранных метрик сравнения.

  \begin{center}
    \includegraphics[width=0.95\columnwidth]{pics/top_n_decay.png}

    \small\emph{Рис. 3. График падения метрики для 5 лучших результатов. Резкий разрыв между первым (истинным) совпадением и последующими результатами подтверждает высокую избирательность алгоритма.}
  \end{center}

  \begin{center}
    \includegraphics[width=0.95\columnwidth]{pics/ramachandran.png}

    \small\emph{Рис. 4. Проверка стереохимической корректности найденного каркаса (1t66). Точки попадают в разрешенные области (альфа-спирали и бета-листы).}
  \end{center}

  \item \textbf{Автономность}: Система полностью независима от внешних веб-сервисов и интернет-соединения после первичной загрузки, что является критическим требованием для работы в закрытых исследовательских контурах и обеспечивает воспроизводимость результатов.
\end{enumerate}

\section*{4. Используемая литература}

\begin{enumerate}
  \item \emph{Abbas A. K., Lichtman A. H., Pillai S.} Cellular and Molecular Immunology. — 9th ed. — Elsevier, 2017.
  \item \emph{Dondelinger M. et al.} Understanding the significance and implications of antibody numbering and antigen-binding surface/residue definition // Frontiers in Immunology. — 2018. — Vol. 9. — P. 2278.
  \item \emph{Dunbar J., Deane C. M.} ANARCI: antigen receptor numbering and receptor classification // Bioinformatics. — 2016. — Vol. 32.
  \item \emph{Edelman G. M.} Antibody Structure and Molecular Immunology: Nobel Lecture. — 1972.
\end{enumerate}

\end{document}
